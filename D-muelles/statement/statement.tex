\documentclass[10.5pt]{article}
\usepackage[margin=1.1in]{geometry}
\usepackage[utf8]{inputenc}
\usepackage{ amssymb }
\usepackage{amsmath}
\usepackage{mdframed}
\usepackage{minted}
\surroundwithmdframed[linewidth = 1pt]{minted}
\usepackage{parskip}
\usepackage{graphicx}
\usepackage{float}
\usepackage{fancyhdr}
\setlength{\headheight}{3cm}
\pagestyle{fancy}
\renewcommand{\headrulewidth}{0pt}


\lhead{
\begin{minipage}{2.5cm}
\begin{figure}[H]
\includegraphics[width=2.5cm]{oie_logo.png}
\end{figure}
\end{minipage} 
\begin{minipage}{8cm}
\textbf{XXV Olimpiada Inform\'atica Espa\~nola}\\
Pruebas de sistema\\
\texttt{muelles}
\end{minipage}
}

\begin{document}



\section*{El espect\'aculo de los muelles}

\'Elater es un gran mago. Su funci\'on m\'as famosa es \textit{El espect\'aculo de los muelles}. Consiste en lo siguiente: 

Hay $n$ muelles el\'asticos en una fila colgados del techo a diferentes alturas. El $i$-\'esimo muelle est\'a colgado a una altura $h_i$ y tiene una \textit{constante de elasticidad} $k_i$. Si colgamos una pesa de masa $w$ en el extremo inferior del $i$-\'esimo muelle, la pesa descender\'a hasta una altura dada por la siguiente f\'ormula:

$$H = h_i - \frac{w}{k_i}\text{.}$$

Durante la funci\'on, \'Elater responder\'a preguntas de la audiencia. Cuando le pregunten por un entero positivo $w$, \'Elater ser\'a capaz de adivinar cu\'al de los $n$ muelles descender\'a a la menor altura (es decir, m\'as cerca del suelo) al colgarse una pesa de masa $w$ de \'el (en caso de que hayan varios muelles que desciendan exactamente a la misma altura, \'Elater puede responder cualquiera de ellos). Para conseguir esto, \'Elater cuenta con la ayuda de su inestimable ayudante, Li Chao.

Antes de la funci\'on, Li Chao tiene algo de tiempo para hacer medidas en los muelles. No puede medir directamente los valores de $h_i$ y $k_i$, pero puede escoger dos muelles $a$ y $b$ y una masa entera $w$, colgar una pesa de masa $w$ de cada uno de los dos muelles $a$ y $b$, y ver cu\'al se acerca m\'as al suelo. Antes de la funci\'on, Li Chao tiene tiempo para hacer $20\,000$ medidas de este tipo. Durante la funci\'on, \'Elater ir\'a recibiendo preguntas con distintos valores de $w$. Despu\'es de cada pregunta, \'Elater puede distraer a la audiencia un rato, de forma que a Li Chao le da tiempo a hacer $20$ medidas m\'as antes de chivarle discretamente cu\'al es la respuesta a \'Elater.

¿Puedes ayudar a Li Chao a hacer que el espect\'aculo sea un \'exito?

\small{
Debes asumir que los muelles est\'an colocados a suficiente altura y los pesos son suficientemente peque\~nos para que los muelles nunca toquen el suelo en ning\'un momento. 

N\'otese que las pesas se quitan de los muelles despu\'es de hacer cada medida.
}

\subsection*{Entrada y salida}

\textbf{Este es un problema interactivo}. Debes refrescar la salida cada vez que imprimas datos (\texttt{cout << endl} o \texttt{cout << flush} en C++, \texttt{System.out.flush()} en Java, \texttt{stdout.flush()} en Python).

La primera l\'inea de la entrada contiene un entero $n$, el n\'umero de muelles. Debes leer este valor antes de hacer ninguna pregunta.

Para hacer una medida debes escribir una l\'inea con el formato \verb#? a b w#, donde $a, b$ son los \'indices de los muelles que quieres usar $0 \leq a, b \leq n-1$ y $w$ es el peso entero que quieres poner en los dos muelles, y debe satisfacer $1 \leq w \leq 10^5$. Despu\'es de hacer una medida, debes leer de la entrada una palabra, el resultado. Esta palabra ser\'a \verb#FIRST# si el muelle $a$ llega a una altura inferior, \texttt{SECOND} si el muelle $b$ llega a una altura inferior, o \texttt{EQUAL} si los dos llegan a la misma altura.

Durante la primera parte de la interacci\'on (correspondiente a las medidas antes de la funci\'on), puedes hacer como m\'aximo $20\,000$ medidas. Despu\'es de hacer todas las medidas de esta primera parte, debes imprimir una l\'inea con el car\'acter \texttt{!}. Despu\'es de esto, recibir\'as una o m\'as preguntas.

Cada pregunta se dar\'a en una l\'inea de la entrada con el formato \texttt{QUESTION w}, donde $w$ es el entero que se pregunta a \'Elater. Despu\'es de leer cada pregunta, puedes hacer como m\'aximo $20$ medidas (con el mismo formato que antes). Cuando hayas hecho las medidas y hayas determinado la respuesta a la pregunta, debes imprimir una l\'inea con el formato \verb#! i#, donde $i$ es el \'indice del muelle que es la respuesta a la pregunta ($1 \leq i \leq n-1$). Si hay varios muelles que descienden a la misma altura m\'inima, puedes responder cualquiera de ellos.

Cuando no haya m\'as preguntas, leer\'as una l\'inea con la palabra \texttt{FINISH}. Despu\'es de leer esto tu programa debe terminar. 

En caso de que en alg\'un momento escribas una pregunta inv\'alida o te pases del l\'imite de preguntas, leer\'as la palabra \texttt{ERROR}. En ese caso tu programa deber\'a terminar inmediatamente.

\subsection*{Ejemplo}

Entrada:
\begin{minted}[obeytabs=true, breaklines, breakautoindent=true]{text}
3
SECOND
QUESTION 2
SECOND
FIRST
QUESTION 6
SECOND
SECOND
FINISH
\end{minted}



Salida:
\begin{minted}[obeytabs=true, breaklines, breakautoindent=true]{text}
? 0 1 1
!
? 0 1 2
? 1 2 2
! 1
? 0 1 6
? 1 2 6
! 2
\end{minted}

\subsection*{Restricciones}

$2 \leq n \leq 500$.

Los pesos de las preguntan satisfacen $1 \leq w \leq 10^5$. Los pesos con los que hagas medidas tambi\'en deben satisfacer $1 \leq w \leq 10^5$.

Se pueden hacer como m\'aximo $20\,000$ medidas durante la primera parte y $20$ medidas despu\'es de cada pregunta.

El n\'umero total de preguntas estar\'a entre $1$ y $1000$. 

\subsection*{Subtareas}

\begin{enumerate}
    \item (20 puntos) $n \leq 100$.
    \item (80 puntos) Sin restricciones adicionales.
    
\end{enumerate}



\end{document}
