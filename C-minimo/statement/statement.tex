\documentclass[10.5pt]{article}
\usepackage[margin=1.1in]{geometry}
\usepackage[utf8]{inputenc}
\usepackage{ amssymb }
\usepackage{amsmath}
\usepackage{mdframed}
\usepackage{minted}
\surroundwithmdframed[linewidth = 1pt]{minted}
\usepackage{parskip}
\usepackage{graphicx}
\usepackage{float}
\usepackage{tikz}
\usepackage{fancyhdr}
\setlength{\headheight}{3cm}
\pagestyle{fancy}
\renewcommand{\headrulewidth}{0pt}


\lhead{
\begin{minipage}{2.5cm}
\begin{figure}[H]
\includegraphics[width=2.5cm]{oie_logo.png}
\end{figure}
\end{minipage} 
\begin{minipage}{8cm}
\textbf{XXV Olimpiada Inform\'atica Espa\~nola}\\
Pruebas de sistema\\
\texttt{minimo}
\end{minipage}
}

\begin{document}



\section*{M\'inimo local}

Dada una lista de $n$ enteros $a_0, \ldots, a_{n-1}$, decimos que una posici\'on $i$ es un m\'inimo local si el elemento $i$-\'esimo es menor que los elementos adyacentes, es decir,  si $a_{i-1} > a_i$ y $a_i < a_{i+1}$ en el caso de que $1 \leq i \leq n-2$, $a_i < a_1$ en el caso $i = 0$ o $a_{n-2} > a_i$ en el caso $i = n-1$. 

Hay una lista de $n$ enteros $a_0, \ldots, a_{n-1}$ de $0$ a $n-1$ sin repeticiones (es decir, una permutaci\'on de los elementos $0, 1, \ldots, n-1$) que no es conocida. Puedes hacer preguntas del siguiente tipo: proporcionas dos \'indices $i$, $j$ y se te responde si $a_i$ es mayor o menor que $a_j$. Debes encontrar un m\'inimo local de esta lista haciendo como m\'aximo $50$ preguntas.



\subsection*{Entrada y salida}

\textbf{Este es un problema interactivo}. Debes refrescar la salida cada vez que imprimas datos (\texttt{cout << endl} o \texttt{cout << flush} en C++, \texttt{System.out.flush()} en Java, \texttt{stdout.flush()} en Python).

La primera l\'inea de la entrada contiene un entero $n$, la longitud de la permutaci\'on. Debes leer este valor antes de hacer ninguna pregunta.

Para hacer una pregunta debes escribir una l\'inea con el formato \verb#? i j#, donde $i, j$ son los \'indices que quieres consultar ($0 \leq i, j \leq n-1$). A continuaci\'on debes leer una l\'inea con el resultado, que ser\'a un car\'acter: \verb#<# si $a_i < a_j$ o \verb#># si $a_i > a_j$. En caso de que hayas superado el l\'imite de consultas o hayas hecho una consulta inv\'alida, leer\'as el car\'acter \verb#-#. Si tu programa lee un \verb#-#, debe terminar inmediatamente.

Para dar el resultado, debes escribir una l\'inea con el formato \verb#! i#, donde $i$ es el \'indice del m\'inimo local. Si hay varios m\'inimos locales puedes imprimir cualquiera de ellos.



\subsection*{Ejemplo}

Entrada:
\begin{minted}[obeytabs=true, breaklines, breakautoindent=true]{text}
3
>
<
\end{minted}

Salida:
\begin{minted}[obeytabs=true, breaklines, breakautoindent=true]{text}
? 0 1
? 1 2
! 1
\end{minted}

Esta interacci\'on se podr\'ia corresponder con las permutaciones $1, 0, 2$ o $2, 0, 1$. En ambos casos, el m\'inimo local est\'a en la posici\'on $1$.

\subsection*{Restricciones}

$2 \leq n \leq 10^6$.

Puedes hacer como mucho $50$ preguntas.

\subsection*{Subtareas}

\begin{enumerate}
    \item (10 puntos) $n \leq 7$.
    \item (15 puntos) $n \leq 50$.
    \item (20 puntos) $n \leq 100$.
    \item (55 puntos) Sin restricciones adicionales.
    
\end{enumerate}



\end{document}
